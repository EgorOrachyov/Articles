\section{Evaluation}

For performance analysis of the proposed solution, an evaluated is conducted of a few most common graph algorithms using real-world sparse matrix data. As a baseline for comparison we chose LAGraph~\cite{misc:la_graph} in connection with SuiteSparse~\cite{article:suite_sparse_for_graph_problems} as a CPU analysis tool, Gunrock~\cite{article:gunrock} and GraphBLAST~\cite{yang2019graphblast} as a Nvidia GPU tools. Also, we tested algorithms on several devices with distinct OpenCL vendors in order to validate the portability of the proposed solution. 

\subsection{Experiments description}

\textbf{Research questions}. In general, this evaluation intentions are summarized in the following research questions

\begin{itemize}
    \item[\textbf{RQ1}] What is the performance of the proposed solution relative to existing tools for both CPU and GPU analysis?
    
    \item[\textbf{RQ2}] What is the portability of the proposed solution with respect to various device vendors and OpenCL runtimes?
\end{itemize}

\textbf{Setup}. For evaluation, we use a PC with Ubuntu 20.04 installed, which has 3.40Hz Intel Core i7-6700 4-core CPU, DDR4 64Gb RAM, Intel HD Graphics 530 integrated GPU, and Nvidia GeForce GTX 1070 dedicated GPU with 8Gb on-board VRAM. Host programs were compiled with GCC v9.3.0. Programs using CUDA were compiled with GCC v8.4.0 and Nvidia NVCC v10.1.243. Release mode and maximum optimization level were enabled for all tested programs. Data loading time, preparation, format transformations, and host-device initial communications are excluded from time measurements. All tests are averaged across 10 runs. Additional warm-up run for each test execution is excluded from measurements.\\

\textbf{Graph algorithms}. For preliminary study \textit{breadth-first search} (BFS) and \textit{triangles counting} (TC) algorithms were chosen, since they allow analyse the performance of \textit{vxm} and \textit{mxm} operations, rely heavily on \textit{masking}, and utilize \textit{reduction} or \textit{assignment}. BFS implementation utilizes automated vector storage sparse-to-dense switch and only \textit{push optimization}. TC implementation uses masked \textit{mxm} of source lower-triangular matrix multiplied by itself with second transposed argument.\\

\textbf{Dataset}. Nine matrices were selected from the Sparse Matrix Collection at University of Florida~\cite{dataset:sparse_matrix_collection}. Information about graphs is summarized in Table~\ref{dataset:info}. All datasets are converted to undirected graphs. Self-loops and duplicated edges are removed.\\

\begin{table}[tbp]
\caption{Dataset description.} 
\begin{center}
    \rowcolors{2}{black!2}{black!10}
    \begin{tabular}{|l|r|r|r|}
    \hline
    Dataset & Vertices  & Edges & Max Degree \\
    \hline
    \hline
    coAuthorsCiteseer & 227.3K &   1.6M &    1372 \\
    coPapersDBLP      & 540.4K &  30.4M &    3299 \\
    hollywood-2009    &   1.1M & 113.8M &  11,467 \\
    roadNet-CA        &   1.9M &   5.5M &      12 \\
    com-Orkut         &     3M &   234M &   33313 \\
    cit-Patents       &   3.7M &  16.5M &     793 \\
    rgg\_n\_2\_22\_s0 &   4.1M &  60.7M &      36 \\
    soc-LiveJournal   &   4.8M &  68.9M &  20,333 \\
    indochina-2004    &   7.5M & 194.1M & 256,425 \\
    \hline
    \end{tabular}
    \label{dataset:info}
\end{center}
\end{table}

\subsection{Results}

\begin{table}[h]
\caption{Breadth-first search algorithm evaluation results.\\Time in milliseconds (lower is better).} 
\begin{center}
    \begin{tabular}{|l|r|r|r|r|r|}
    \hline
    \multirow{2}{*}{Dataset} & \multicolumn{3}{c|}{Nvidia} & \multicolumn{2}{c|}{Intel} \\
    \cline{2-6}
    & GR & GB & SP & SS & SP \\
    \hline
    \hline
    \rowcolor{black!10} hollywood-2009    &  20.3 &  82.3 &   36.9 &   23.7 &   303.4 \\
    \rowcolor{black!2 } roadNet-CA        &  33.4 & 130.8 & 1456.4 &  168.2 &   965.6 \\
    \rowcolor{black!10} soc-LiveJournal   &  60.9 &  80.6 &   90.6 &   75.2 &  1206.3 \\
    \rowcolor{black!2 } rgg\_n\_2\_22\_s0 &  98.7 & 414.9 & 4504.3 & 1215.7 & 15630.1 \\
    \rowcolor{black!10} com-Orkut         & 205.2 & -- -- &  117.9 &   43.2 &   903.6 \\
    \rowcolor{black!2 } indochina-2004    &  32.7 & -- -- &  199.6 &  227.1 &  2704.6 \\
    \hline
    \hline
    \multicolumn{6}{l}{Tools: Gunrock (GR), GraphBLAST (GB), SuiteSparse (SS), Spla (SP).} \\
    \end{tabular}
    \label{results:bfs}
\end{center}
\end{table}

\begin{table}[h]
\caption{Triangles counting algorithm evaluation results.\\Time in milliseconds (lower is better).} 
\begin{center}
    \begin{tabular}{|l|r|r|r|r|r|}
    \hline
    \multirow{2}{*}{Dataset} & \multicolumn{3}{c|}{Nvidia} & \multicolumn{2}{c|}{Intel} \\
    \cline{2-6}
    & GR & GB & SP & SS & SP \\
    \hline
    \hline
    \rowcolor{black!10} coAuthorsCiteseer &   2.1 &    2.0 &    9.5 &    17.5 &    64.9 \\
    \rowcolor{black!2 } coPapersDBLP      &   5.7 &   94.4 &  201.9 &   543.1 &  1537.8 \\
    \rowcolor{black!10} roadNet-CA        &  34.3 &    5.8 &   16.1 &    47.1 &   357.6 \\
    \rowcolor{black!2 } com-Orkut         & 218.1 & 1583.8 & 2407.4 & 23731.4 & 15049.5 \\
    \rowcolor{black!10} cit-Patents       &  49.7 &   52.9 &   90.6 &   698.3 &   684.1 \\
    \rowcolor{black!2 } soc-LiveJournal   &  69.1 &  449.6 &  673.9 &  4002.6 &  3823.9 \\
    \hline
    \hline
    \multicolumn{6}{l}{Tools: Gunrock (GR), GraphBLAST (GB), SuiteSparse (SS), Spla (SP).} \\
    \end{tabular}
    \label{results:tc}
\end{center}
\end{table}

Tables~\ref{results:bfs} and~\ref{results:tc} present results of the evaluation and compare the performance of Spla (proposed library) against other tools on different execution platforms. Tools are grouped by the type of device for the execution, where either Nvidia or Intel device is used. Cell left empty if tested tool failed to analyze graph due to \textit{out of memory} exception.\\

\textbf{RQ1} \textit{What is the performance of the proposed solution relative to existing tools for both CPU and GPU analysis?}

In general, Spla BFS shows acceptable performance, especially on graphs with large vertex degrees, such as soc-LiveJournal and com-Orkut. On graphs roadNet-CA and rgg it has a significant performance drop due to the nature of underlying algorithms and data structures. Firstly, the library utilizes immutable data buffers. Thus, iteratively updated dense vector of reached vertices must be copied for each modification, which dominates the performance of the library on a graph with a large search depth. Secondly, Spla BFS does not utilize \textit{pull optimization}, which is critical in a graph with a relatively small search frontier and with a large number of reached vertices. 

Spla TC has a good performance on GPU, which is better in all cases than reference SuiteSparse solution. But in most tests GPU competitors, especially Gunrock, show smaller processing times. GraphBLAST shows better performance as well. The library utilizes a masked SpGEMM algorithm, the same as in GraphBLAST, but without \textit{identity} element to fill gaps. Library explicitly stores all non-zero elements, and uses mask to reduce only non-zeros while evaluating dot products of rows and columns. What causes extra divergence inside work groups. 

Gunrock shows nearly the best average performance due to its specialized and optimized algorithms. Also, it has good time characteristics on a mentioned earlier roadNet-CA and rgg in BFS algorithm. GraphBLAST follows Gunrock and shows good performance as well.  But it runs out of memory on two significantly large graphs con-Orkut and indochina-2004. Spla does not rut out of memory on any test due to the simplified storage scheme.\\

\textbf{RQ2} \textit{What is the portability of the proposed solution with respect to various device vendors and OpenCL runtimes?}

On a Nvidia device Spla algorithms' performace in general is acceptable. But it still slower than its competitors, such as Gunrock or GraphBLAST. However, in both BFS and TC the performance gap is maintained in a predictable fashion.

Spla BFS algorithm suffers a lot on a Intel device compared to SuiteSparse implementation. This is caused due to intense data access and relatively small computations parallelism though traversal. On-chip memory has larger latency, so CPU optimized solution shows better results. 

However, on Intel device Spla TC algorithm shows better performance compared to SuiteSparse on com-Orkut, cit-Patents, and soc-LiveJournal. A possible reason is the large lengths of processed rows and columns in the product of matrices. So, even embedded GPUs can improve the performance of graph analysis in some cases.\\

\textbf{Summary}. Evaluation of proposed solution for real-world graph analysis allows to conclude, that initial OpenCL-based operations implementation with a limited set of optimizations has promising performance compared to other tools and can be easily executed on devices of multiple vendors, which gives significant flexibility in a choice of HPC hardware. 