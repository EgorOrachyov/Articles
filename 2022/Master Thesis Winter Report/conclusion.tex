\section{Results}

The following results were achieved in this work.

\begin{itemize}
    \item The survey of the field was conducted. Model for a graph analysis were shown. Also, the concept of the linear algebra based approach was described in a great detail with a respect to a graph traversal and existing solutions. Introduction into a GraphBLAS standard was provided. Existing implementations, frameworks and most significant contributions for a graph analytic were studied. Their limitations were highlighted. 
    
    \item General-purpose GPU computations concept was covered. Different APIs for GPU programming were presented. Their advantages and disadvantages are covered. General GPU programming challenges and pitfalls are highlighted. 
    
    \item The architecture of the library for a generalized sparse linear algebra for GPU computations was developed. The architecture and library design was based on a project requirements, as well as on a limitation and experience of the existing solutions. 
    
    \item The implementation of the library accordingly to the developed architecture was started. The core of the library, expressions processing, foundation OpenCL functionality, common operations implementations were provided.
    
    \item Several algorithms for a graph analysis were implemented using developed library API.
    
    \item The preliminary experimental study of the proposed artifacts was conducted. Obtained results allowed to conclude, that the chosen method of the library development is a promising way to a high-performance graph analysis in terms of the linear algebra on a wide family of GPU devices. 
\end{itemize}

The following tasks must be done to complete this work.

\begin{itemize}
    \item Extend a set of available linear algebra operations, implemented in the library.

    \item Implement a set of a common graph analysis algorithms utilising library primitives and operations, as well as introducing some optimizations  for this algorithms.
    
    \item Conduct a complete experimental study of the set of common graph analysis algorithms. Extend the dataset and study the edge cases of library workarounds.
\end{itemize}

The library source code is published on a GitHub platform. It is available at \url{https://github.com/JetBrains-Research/spla}.
