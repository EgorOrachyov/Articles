\section{Экспериментальное исследование}

\subsection{Постановка экспериментов}

Для экспериментов использовалась рабочая станция с процессором Intel Core i7-6790, тактовой частотой 3.40GHz, RAM DDR4 с объемом памяти 64Gb, видеокартой GeForce GTX 1070 с 8Gb VRAM, ОС под управлением Ubuntu 20.04.

Данные, необходимые для замеров, предварительно загружаются в RAM или VRAM в формате, требуемом для тестируемого интсрумента. Время, необходимое на чтение данных с диска, их конвертацию, а также подготовку начального состояния входных матриц исключено из замеров. 

\subsubsection*{Исследовательские вопросы}

Для того, чтобы стуктурировать исследование, были сформулированы следующие вопросы.

\begin{itemize}
   \item[\textbf{В1:}] Какова производительность отдельных операций реализвонной библиотеки примитивов разреженной линейной булевой алгебры на GPGPU по сравнению с существующими аналогами?
   
   \item[\textbf{В2:}] Какова производиетльность реализованного алгоритма поиска путей через тензорное произведение на GPGPU  по сравнению с существующими аналагами, также полагающимися на примтивы линейной алгебры? 
\end{itemize}

В1 направлен на определение эффективности отдельных матрицных операций в реализованной библиотеке. В качестве таких операций выступают \textit{матричное умножение} и \textit{матрицное сложение} в булевом полукольце, как наиболее распространненные и критически важные операции в прикладных алгоритмах. Для сравнения производительности в этих операциях предлагается использовать популярные существующие библиотеки разреженной линейной алгебры для платформ Nvidia Cuda, OpenCL и CPU. В качетсве таких библиотек ьыли выбраны были выбраны CUSP и cuSPRASE для Nvidia Cuda, clSPARSE для OpenCL, и SuiteSparse для CPU. CUSP предаставляет реализацию операций, основанную на шаблонах для параметризации используемого типа данных, однако библиотека не делает каких-либо дополнительных оптимизация конкретно для булевых значений. cuSPARSE и clSPRASE предоставляют операции только для основных типов данных с плавающей запятой. Однако данное ограничение можно обойти, если интерпретировать ненулевые значения как \textit{true}. Библиотека SuiteSprase является эталонной реализацией GraphBLAS API и имеет встроенное булево полукольцо для вычислений.

В2 направлен на определение эффективности реализованного алгоритма и его сравнение с алгоритмом Рустама Азимова~\cite{inproceedings:cfqp_matrix_with_single_source}, который также полагается на операции линейной булевой алгебры. Данный алгоритм также реализован с использованием разработанного в данной работе Python-пакета, что делает сравнение корректным.

\subsubsection*{Набор данных}

Для замеров производительности отдельных операций реализованной библиотеки были выбраны 10 различных квадратных матриц из известной коллекции университета Флориды~\cite{net:sp_matrix_data_florida} для проверки эффективности алгоритмов, реализующих операции над разряженными матрицами. 
Информация о матрицах представлена в таблице~\ref{table:sparse_matrices}. 
Для обозначения числа ненулевых элементов используется аббревиатура \textit{Nnz} (англ. number of non-zero elements). 
В таблице приведено официальное название матрицы, количество строк (соответсвует числу столбцов), а также количество ненулевых элментов в данной матрице и в производных от нее, полученных умножением матрицы самой на себя, что обозначается как степень $M^2$, и поэлементным сложением данной матрицы с собой также возведенной в степень, что обозначается как $M + M^2$.
Вычисление данных артефатков имитирует шаг транзитивного замыкания.
Эффективное вычисление этого шага во многом определяет производительность конечных пользовательских алгоритмов на графах.

\begin{table}[t]
\begin{center}
\caption{Разреженные матричные данные}
\label{table:sparse_matrices}
% \scriptsize
\rowcolors{2}{black!2}{black!10}
\scalebox{0.8}{
\begin{tabular}{|l|r|r|r|r|}
\hline
Матрица $M$      & Кол-во Строк & Nnz $M$      & Nnz   $M^2$    & Nnz $M + M^2$    \\
\hline
\hline
wing             &    62,032    &   243,088    &    714,200     &    917,178       \\
luxembourg\_osm  &   114,599    &   239,332    &    393,261     &    632,185       \\
amazon0312       &   400,727    & 3,200,400    & 14,390,544     & 14,968,909       \\
amazon-2008      &   735,323    & 5,158,388    & 25,366,745     & 26,402,678       \\
web-Google       &   916,428    & 5,105,039    & 29,710,164     & 30,811,855       \\
roadNet-PA       & 1,090,920    & 3,083,796    &  7,238,920     &  9,931,528       \\
roadNet-TX       & 1,393,383    & 3,843,320    &  8,903,897     & 12,264,987       \\
belgium\_osm     & 1,441,295    & 3,099,940    &  5,323,073     &  8,408,599       \\
roadNet-CA       & 1,971,281    & 5,533,214    & 12,908,450     & 17,743,342       \\
netherlands\_osm & 2,216,688    & 4,882,476    &  8,755,758     & 13,626,132       \\ 
\hline
\end{tabular}
}
\end{center}
\end{table}

Для замеров производительности алгоритмов поиска путей с КС ограничениями используется коллекция графовых данных лаборатории языковых интсрументов JetBrains Research~\cite{net:cfpq_data}, которая использовались в ряде работ~\cite{inproceedings:matrix_cfpq, inproceedings:cfpq_matrix_evaluation, inbook:kronecker_cfpq_adbis, inproceedings:cfqp_matrix_with_single_source} для подобных экспериментов. Данная колекция содержит RDF данные, онтологии, графы программ для анализа указателей, а также ряд сгенерированных графов для анализа особых случаем. 

\subsubsection*{Метрики}

Для ответа на поставленные исследовательские вопросы в качестве метрик производительности используется время, требуемое для выполнения операции, а также пиковое количество потребяемой видеопамяти на GPU в момент вычисления. Показатели времени усреднены по 10 запускам. Предварительно совершался не учитывающийся в замерах запуск, чтобы проинициализировать начальное состояние тестируемых библиотек. Показатели потребления видеопамяти получены с помощью интсрумента \textit{nvidia-smi}, который с точностью до $1$ миллисикунды позволяет отслеживать количество потребляемой памяти процессом ОС на стороне видеокарты. 

\subsection{Результаты}


\textit{1) В1: Какова производительность отдельных операций реализвонной библиотеки примитивов разреженной линейной булевой алгебры на GPGPU по сравнению с существующими аналогами?} 

Результаты эксперимента по сравнению производительности матричного произведения представлены в таблице~\ref{table:eval_mm_results}.
Реализованная библиотека cuBool показывает лучшие результаты по сравнению с другими библиотеками. 
Используемы в реализации алгоритм Nsparse позволяет получить прирост в скорости до 5 раз, а также сократить потребление видеопамяти до 8 раз, что особенно заметно в сравнении с такими библиотеками как CUSP или clSPARSE.

Результаты эксперимента по сравнению производительности матричного поэлементного сложения представлены в таблице~\ref{table:eval_ma_results}. 
Библиотека clSPRARSE не реализует данную операцию, поэтому относящаяся к ней колонка с результатами оставлена пустой.
cuBool демонстрируют хорошую производительность, его показатели времени сравнимы с такими промышленными библиотеками как CUSP или cuSPRASE и отличаются незначительно как в большую, так и меньшую сторону. 
Однако используемая cuBool операция сложения потребляет значительно меньше видеопамяти во время обработки, что позволяет местами достигать до 3 раз меньших значений в сравнении с CUSP.

\textit{2) В2: Какова производиетльность реализованного алгоритма поиска путей через тензорное произведение на GPGPU по сравнению с существующими аналагами, также полагающимися на примтивы линейной алгебры?}

\begin{table}[]
\begin{center}
\caption{Матричное умножение (время (t) в миллисекундах, память (m) в мегабайтах)}
\label{table:eval_mm_results}
% \scriptsize
\rowcolors{4}{black!2}{black!10}
\scalebox{0.8}{
\begin{tabular}{| l | r r | r r | r r | r r | r r |}
\hline
Матрица $М$ & \multicolumn{2}{c|}{cuBool} & \multicolumn{2}{c|}{CUSP} & \multicolumn{2}{c|}{cuSPRS} & \multicolumn{2}{c|}{clSPRS} & \multicolumn{2}{c|}{SuiteSprs} \\   
                  & t    & m   & t     & m    & t      & m   & t     & m    & t     & m   \\
\hline
\hline
 wing             & 1.9  & 93  & 5.2   & 125  & 20.1   & 155 & 4.2   & 105  & 7.9   & 22  \\ % 1.  wing             
 luxembourg\_osm  & 2.4  & 91  & 3.7   & 111  & 1.7    & 151 & 6.9   & 97   & 3.1   & 169 \\ % 2.  luxembourg\_osm  
 amazon0312       & 23.2 & 165 & 108.5 & 897  & 412.8  & 301 & 52.2  & 459  & 257.6 & 283 \\ % 3.  amazon0312       
 amazon-2008      & 33.3 & 225 & 172.0 & 1409 & 184.8  & 407 & 77.4  & 701  & 369.5 & 319 \\ % 4.  amazon-2008      
 web-Google       & 41.8 & 241 & 246.2 & 1717 & 4761.3 & 439 & 207.5 & 1085 & 673.3 & 318 \\ % 5.  web-Google       
 roadNet-PA       & 18.1 & 157 & 42.1  & 481  & 37.5   & 247 & 56.6  & 283  & 66.6  & 294 \\ % 6.  roadNet-PA       
 roadNet-TX       & 22.6 & 167 & 53.1  & 581  & 46.7   & 271 & 70.4  & 329  & 80.7  & 328 \\ % 7.  roadNet-TX       
 belgium\_osm     & 23.2 & 151 & 32.9  & 397  & 26.7   & 235 & 68.2  & 259  & 56.9  & 302 \\ % 8.  belgium\_osm     
 roadNet-CA       & 32.0 & 199 & 74.4  & 771  & 65.8   & 325 & 98.2  & 433  & 114.5 & 344 \\ % 9.  roadNet-CA       
 netherlands\_osm & 35.3 & 191 & 51.0  & 585  & 51.4   & 291 & 102.8 & 361  & 90.9  & 311 \\ % 10. netherlands\_osm  
\hline
\end{tabular}
}
\end{center}
\end{table}

\begin{table}[]
\begin{center}
\caption{Поэлементное матричное сложение (время (t) в миллисекундах, память (m) в мегабайтах)}
\label{table:eval_ma_results}
% \scriptsize
\rowcolors{4}{black!2}{black!10}
\scalebox{0.8}{
\begin{tabular}{| l | r r| r r | r r | r r | r r |}
\hline
Матрица $М$ & \multicolumn{2}{c|}{cuBool} & \multicolumn{2}{c|}{CUSP} & \multicolumn{2}{c|}{cuSPRS} & \multicolumn{2}{c|}{clSPRS} & \multicolumn{2}{c|}{SuiteSprs} \\   
                  & t    & m   & t    & m   & t    & m   & t      & m      & t    & m   \\
\hline
\hline
 wing             & 1.1  & 95  & 1.4  & 105 & 2.4  & 163 & ~~~~-  & ~~~~-  & 2.3  & 176 \\ % 1.  wing             
 luxembourg\_osm  & 1.7  & 95  & 1.0  & 97  & 0.8  & 151 & ~~~~-  & ~~~~-  & 1.6  & 174 \\ % 2.  luxembourg\_osm  
 amazon0312       & 11.4 & 221 & 16.2 & 455 & 24.3 & 405 & ~~~~-  & ~~~~-  & 37.2 & 297 \\ % 3.  amazon0312       
 amazon-2008      & 17.5 & 323 & 29.5 & 723 & 27.2 & 595 & ~~~~-  & ~~~~-  & 64.8 & 319 \\ % 4.  amazon-2008      
 web-Google       & 24.8 & 355 & 31.9 & 815 & 89.0 & 659 & ~~~~-  & ~~~~-  & 77.2 & 318 \\ % 5.  web-Google       
 roadNet-PA       & 16.9 & 189 & 11.2 & 329 & 11.6 & 317 & ~~~~-  & ~~~~-  & 36.6 & 287 \\ % 6.  roadNet-PA       
 roadNet-TX       & 19.6 & 209 & 14.5 & 385 & 16.9 & 357 & ~~~~-  & ~~~~-  & 45.3 & 319 \\ % 7.  roadNet-TX       
 belgium\_osm     & 19.5 & 179 & 10.2 & 303 & 10.5 & 297 & ~~~~-  & ~~~~-  & 28.5 & 302 \\ % 8.  belgium\_osm     
 roadNet-CA       & 30.5 & 259 & 19.4 & 513 & 20.2 & 447 & ~~~~-  & ~~~~-  & 65.2 & 331 \\ % 9.  roadNet-CA       
 netherlands\_osm & 30.1 & 233 & 14.8 & 423 & 18.3 & 385 & ~~~~-  & ~~~~-  & 50.2 & 311 \\ % 10. netherlands\_osm  
\hline
\end{tabular}
}
\end{center}
\end{table}

