\section{Детали реализации}

В данной секции предлагается рассмотреть основные детали реализации библиотеки \textit{cuBool} и соответствующего ей Python-пакета \textit{pycubool}, которые были разработаны в соответствии с архитектурой, представленной в предыдущей главе.

Разработка библиотеки осуществлялась в рамках исследовательского проекта лаборатории языковых инструментов JetBrains Research. В качестве языка программирования для реализации библиотеки используется С++, 
так как он предоставляет механизмы для ручного управления ресурсами, 
а также позволяет использовать язык Cuda C/C++ в рамках единого компилируемого приложения. 
Интерфейс библиотеки реализован в виде C-совместимого API.
Исходный код компилируется в библиотеку с разделяемым кодом \textbf{libcubool.so}, 
которая может быть динамически загружена в конечное пользовательское приложение. 
В качестве целевой платформы для исполнения поддерживается семейство операционных систем на базе ядра Linux.

\subsection{Примитивы и операции}

Основным примитивом библиотеки является разреженная матрица булевых значений, 
которая хранится в видеопамяти видеокарты в формате \textit{CSR} (compressed sparse row), 
который позволяет использовать $O(V + E)$ памяти для хранения матрицы смежности графа. 
Существуют и другие форматы хранения разреженных матриц: \textit{CSC} (compressed sparse column), \textit{COO} (coordinate list) и так далее. 
Однако CSR формат был выбран на основе результатов исследования Юсуке Нагасака и др.~\cite{inproceedings:spgemm_mem_saving_for_nvidia}, 
так как он позволяет эффективно реализовать операцию матричного умножения в условиях ограниченного объема доступной видеопамяти. 

В качестве поэлементных операций сложения и умножения используются \textit{логическое-или} и \textit{логическое-и}. 
Основные функции работы с матрицами представлены ниже.

\begin{itemize}[noitemsep,topsep=0pt,parsep=0pt,partopsep=0pt]
    \item Создание матрицы $M$ размера $m \times n$.
    \item Удаление матрицы $M$ и освобождение занятых ею ресурсов.
    \item Заполнение матрицы $M$ списком значений $\{(i, j)_k\}_k$.
    \item Чтение из матрицы $M$ списка значений $\{(i, j)~|~M[i,j]=1\}$.
    \item Транспонирование матрицы $M = N^T$.
    \item Извлечение подматрицы $M = N[i:k, j:t]$.
    \item Редуцирование матрицы к вектору $V$:~$V[i]=\bigoplus_j M[i,j]$.
    \item Сложение матриц $C \mathrel{+}= M$.
    \item Умножение матриц $C \mathrel{+}= M \times N$.
    \item Произведение Кронекера для двух матриц $C = M \otimes N$.
\end{itemize}

\subsection{Cuda-модуль}

Операции линейной алгебры для работы с матрицами реализованы с использованием технологии Cuda. 
В качестве основы для реализации операций умножения и сложения разреженных матриц используются результаты исследования Арсения Терехова и др.~\cite{inproceedings:cfqp_matrix_with_single_source}, оформленные в виде библиотеки \textbf{Nsparse}.
Данная библиотека была доработана, чтобы добавить возможность динамически конфигурировать механизмы использования видеопамяти. 

Для реализации произведения Кронекера, операций транспонирования, редуцирования и извлечения подматрицы использовались примитивы библиотеки \textbf{Thrust}.
Данная библиотека позволяет оперировать данными в терминах высокоуровневых операций \textit{свертки}, \textit{отображения} и \textit{префиксной суммы}~\cite{net:cuda_thrust}, которые выполняются на графическом процессоре. 
\textbf{Thrust} поставляется совместно с инструментами Cuda-разработки и не требует настройки дополнительных зависимостей.

\subsection{Python-пакет}

Все примитивы и операции библиотеки cuBool доступны внутри Python-пакета pycubool.
Для публикации пакета используется стандартная инфраструктура PyPI.
Вызов функций из \textbf{cuBool C API}, находящихся в скомпилированной библиотеке \textbf{libcubool.so}, осуществляется с помощью модуля \textbf{Ctypes}. 
Данный модуль поставляется вместе с инфраструктурой Python и не требует настройки сторонних зависимостей. 
Также в пакет pycubool добавлены дополнительные операции, которые облегчают использование данного инструмента и предоставляют конечному пользователю дополнительную функциональность.

\begin{itemize}[noitemsep,topsep=0pt,parsep=0pt,partopsep=0pt]
    \item Загрузка и сохранение матрицы в \textit{Matrix market} формате.
    \item Экспортирование набора матриц в \textit{GraphViz} формате.
    \item \textit{Красивая} печать матриц в текстовом виде.
    \item Текстовые маркеры и имена матриц для отладки.
\end{itemize}

\subsection{Пример использования}

\definecolor{codegreen}{rgb}{0,0.6,0}
\definecolor{codegray}{rgb}{0.5,0.5,0.5}
\definecolor{codepurple}{rgb}{0.58,0,0.82}
\definecolor{backcolour}{rgb}{1.0,1.0,1.0}

\lstdefinestyle{codelistingstyle}{
    backgroundcolor=\color{backcolour},   
    commentstyle=\color{codegreen},
    keywordstyle=\color{magenta},
    numberstyle=\tiny\color{codegray},
    stringstyle=\color{codepurple},
    basicstyle=\ttfamily\footnotesize,
    breakatwhitespace=false,         
    breaklines=true,                 
    captionpos=b,                    
    keepspaces=true,                 
    numbers=left,                    
    numbersep=5pt,                  
    showspaces=false,                
    showstringspaces=false,
    showtabs=false,                  
    tabsize=2
}

\lstset{style=codelistingstyle}

\begin{algorithm}[]
\floatname{algorithm}{Listing}
\caption{Пример вычисления транзитивного замыкания с использованием cuBool C API}
\label{alg:cubool_example}
\begin{lstlisting}[language=C++]
#include <cubool/cubool.h>

cuBool_Status TransitiveClosure(cuBool_Matrix A, cuBool_Matrix* T) {
    cuBool_Matrix_Duplicate(A, T);                       /* Копируем матрицу смежности А */

    cuBool_Index total = 0;
    cuBool_Index current;
    cuBool_Matrix_Nvals(*T, &current);                   /* Количество ненулевых значений */

    while (current != total) {                           /* Пока результат меняется  */
        total = current;
        cuBool_MxM(*T, *T, *T, CUBOOL_HINT_ACCUMULATE);  /* T += T x T */
        cuBool_Matrix_Nvals(*T, &current);
    }

    return CUBOOL_STATUS_SUCCESS;
}
\end{lstlisting}
\end{algorithm}

\begin{algorithm}[]
\floatname{algorithm}{Listing}
\caption{Пример вычисления транзитивного замыкания с использованием пакета pycubool}
\label{alg:pycubool_example}
\begin{lstlisting}[language=Python]
import pycubool

def transitive_closure(a: pycubool.Matrix):
    t = a.duplicate()                     # Копируем матрицу смежности А
    total = 0                             # Количество ненулевых значений результата

    while total != t.nvals:               # Пока результат меняется
        total = t.nvals
        t.mxm(t, out=t, accumulate=True)  # t += t x t

    return t
\end{lstlisting}
\end{algorithm}

В качестве примера рассмотрим проблему вычисления \textit{транзитивного замыкания} (англ. transitive closure) для некоторого ориентированного графа без меток $\mathcal{G} = \langle V, E \rangle$. Результатом вычисления транзитивного замыкания является новый граф $\mathcal{G}_{tc} = \langle V, E_{tc} \rangle$, для которого верно следующее: $e = (v,u) \in E_{tc} \iff \exists v \pi u $ в $\mathcal{G}$. Данную проблему можно решить в терминах линейной алгебры, если представить граф в виде матрицы смежности с булевыми значениями. 

В листинге~\ref{alg:cubool_example} представлен фрагмент кода на языке C, который решает данную задачу. В качестве аргументов функция принимает матрицу смежности исходного графа, а также указатель на идентификатор, который необходимо использовать при сохранении результирующей матрицы смежности графа после транзитивного замыкания.

В листинге~\ref{alg:pycubool_example} представлен похожий фрагмент кода, однако он уже решает поставленную в задачу на языке Python. Здесь в качестве входного аргумента используется матрица смежности графа, в качестве результата возвращается матрица смежности графа после транзитивного замыкания.
