\section{Заключение}

В рамках выполнения данной работы были получены следующие результаты:

\begin{itemize}
    \item Спроектирована библиотека примитивов линейной булевой алгебры для работы с разреженными данными на GPGPU. Данная библиотека экспортирует С-совместимый интерфейс, имеет поддержку различных вычислительных модулей, а также предоставляет модуль для работы конечного пользователя с примитивами библиотеки в высокоуровневой среде вычислений с управляемыми ресурсами.

    \item Реализована библиотека cuBool в соответствии с разработанной архитектурой. Ядро библиотеки написано на языке С++, а математические операции, выполняющиеся на GPGPU, реализованы на языке CUDA C/C++. Библиотека предоставляет модуль CPU вычислений для компьютеров без Cuda девайсов. Также создан Python-пакет pycubool, который позволяет использовать функциональность библиотеки в среде Python. Данный пакет доступен для скачивания через пакетный менеджер PyPI.
    
    \item Реализован алгоритм поиска путей с КС ограничениями через тензорное произведение с использованием Python пакета разработанной библиотеки. Данный алгоритм использует операции матричного умножения, сложения и произведение Кронекера в булевом полукольце, а также различные операции для манипуляций над значениями матриц. На вход алгоритм получается представление графа и КС грамматики в виде набора матриц, а на выходе --- возвращает матрицу смежности графа достижимости, а также индекс, который позволяет восстанавливать все пути в графе, в соответствии с входной грамматикой.
    
    \item Выполнено экспериментальное исследование реализованной библиотеки с использованием синтетических и реальных данных из коллекции Разреженных Матриц Университета Флориды. Замеры производительности операции матричного умножения показывают ускорение до 5 раз в сравнении с существующими аналогами при меньшем потреблении памяти. Матричное умножение сравнимо по времени с существующими аналогами, однако потребляет меньше памяти. Также выполнено экспериментальное исследование производительности реализованного алгоритма и его сравнение с существующими алгоритмами для выполнения КС запрос. В качестве данных для замеров использовалась коллекция RDF и синтетических данных Лаборатории Языковых Инструментов JetBrains Research. Замеры показали что (этого мы пока не знаем).
\end{itemize}

На основе результатов, полученных в данном иследовании, была написана статья, принятая на конференцию GrAPL 2021\footnote{GrAPL 2021: Workshop on Graphs, Architectures, Programming, and Learning. Дата обращения: 1.04.2021. Сайт конференции: \url{https://hpc.pnl.gov/grapl/}.}.

Библиотека cuBool и Python-пакет для работы с данной библиотекой доступны для скачивания через следующие онлайн ресурсы: \url{https://github.com/JetBrains-Research/cuBool} и \url{https://test.pypi.org/project/pycubool/}.