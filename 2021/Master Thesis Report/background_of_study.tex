\section{Обзор предметной области}

Для разработки библиотеки необходимо сперва рассмотреть базовую теорию, а также ознакомиться с существующими подходами к реализации. Для этого предлагается ознакомиться с концепциями, предлагаемыми GraphBLAS, а также рассмотреть существующие инструменты для работы с примитивами разреженной линейной алгебры на GPU, а также те проблемы, которые возникают при программировании подобных решений в GPU-среде. Это поможет обосновать необходимость разработки нового инструмента.

\subsection{Концепции GraphBLAS}

Стандарт GraphBLAS~\cite{paper:graphblas_foundations} представляет математическую нотацию, транслированную в некоторый C API. Данный стандарт оперирует концепциями линейной (разреженной) алгебры. Основные элементы данного стандарта изложены ниже.

\begin{itemize}
    \item \textbf{Примитивы или контейнеры.} Основными контейнерами для хранения данных являются \textit{матрица}, \textit{вектор}, \textit{скаляр} и \textit{маска}. Контейнеры параметризуются типом элементов, которые они хранят. Имеется возможность создания пользовательских типов данных. 
    \item \textbf{Алгебраические структуры.} В качестве основных алгебраических структур используются \textit{полукольцо} и \textit{моноид}. Данные структуры адаптированы для разреженных данных, поэтому они отличаются от тех, которые приняты в алгебре.
    Данные структуры определяют поэлементные функции, которые оперируют данными в контейнерах. Например, эти функции используются в качестве параметров \textit{mult} и \textit{add} при вычислении матричного произведения, где элементы строки и столбца сначала умножаются, а затем некоторым образом суммируются в конечный  элемент. 
    \item \textbf{Операции.} Основными операциями являются произведения матриц и векторов, поэлементные операции, транспонирование, операции свертки, применение масок для фильтрации элементов, а также операции для манипулирования данными.
\end{itemize}

\subsection{Существующие инструменты}

Существующие инструменты для анализа графов в терминах разреженной линейной алгебры можно разделить на две основные категории: специализированные библиотеки и библиотеки примитивов разреженной линейно алгебры (без фокуса на анализ графовых данных). Существующие инструменты используют разные типы вычислителей, включая один или несколько центральных процессоров, или GPU-ускоритель. Далее представленные наиболее популярные и влиятельные из них.

\subsubsection*{SuiteSparse}

Библиотека SuiteSparse~\cite{article:suite_sparse_for_graph_problems} является полной и эталонной реализацией стандарта GpraphBLAS для вычислений на центральном процессоре. Она доступна для использования через упоминаемый ранее GraphBLAS C API для программирования в языковой среде C или C++, а также имеет ряд открытых и поддерживаемых сообществом пакетов, для использования функций библиотеки в других языковых средах, таких как Python через pygraphblas~\cite{net:pygraphblas}. 

В проекте ведется активная работа по поддержке технологии Nvidia Cuda для осуществления вычислений на Nvidia GPU-устройстве, однако информации о поддержке нескольких GPU-устройств в одном вычислителе нет. Кроме этого, также остается открытым вопрос о возможности создания собственных пользовательских типов и функций для вычислений на GPU, поскольку Nvidia Cuda API требует предкомпиляции кода, что на данный момент не предусмотрено в стандарте GraphBLAS.  

\subsubsection*{GraphBLAST}

Библиотека GraphBLAST~\cite{yang2019graphblast} предоставляет примтивы и операции разреженной линейной алгебры для вычислений на GPU-устройстве с использованием технологии Nvidia Cuda. Данная библиотека использует схожие с GraphBLAS концепции, однако она имеет C++ интерфейс и использует обобщенное программирование с использованием Cuda C++ шаблонов для обеспечения возможности создания произвольных пользовательских типов и операций.

Использование подобного API оправдано, так как это упрощает написание прикладных алгоритмов и позволяет использовать компилятор для генерации требуемого вспомогательного кода. Однако существенным недостатком такого подхода является то, что весь код содержится в \textit{заголовочных файлах}, что требует от пользователя постоянное наличие Nvidia NVCC компилятора и полную перекомпиляцию приложения при любых модификациях.

На данный момент проект находится на стадии разработки, и, несмотря на презентацию на тематической конференции, часть функциональности проекта еще недоступна.

\subsubsection*{Cusp}

Библиотека cusp~\cite{net:cusplibrary} предоставляет набор примитивов разреженной линейной алгебры и основных операций для вычислений на центральном процессоре (в однопоточном или параллельном режиме) и на Nvidia GPU-устройстве. Библиотека имеет C++ интерфейс на основе шаблонов, использует обобщенное программирование для параметризации операций. В своей основе она использует библиотеку примитивов для параллельного программирования Nvidia Thrust~\cite{net:cuda_thrust}, которая предоставляет реализацию для таких операций, как \textit{sort}, \textit{reduce}, \textit{gather}, \textit{scan} и т.д. 

Cusp имеет сходный с GraphBLAST интерфейс, однако данная библиотека была спроектирована для произвольных вычислений с использованием линейной алгебры без акцента на графовых данных. Поэтому она не поддерживает ряд важных операций, таких как применение маски или редуцирование значений. 

\subsubsection*{cuBool и SPbLA}

Библиотеки cuBool и SPbLA~\cite{article:spbla} являются попыткой реализации примитивов и операций из стандарта GraphBLAS, но только для булевых значений. Существует множество алгоритмов, которые можно выразить с использованием операций булевой разреженной линейной алгебры,
такие как достижимость в графе с регулярными или контекстно-свободными ограничениями~\cite{inproceedings:cfpq_matrix_evaluation, inbook:kronecker_cfpq_adbis, inproceedings:matrix_cfpq, inproceedings:cfqp_matrix_with_single_source}.

Библиотеки используют Nvidia CUDA и OpenCL API для вычислений на GPU. Кроме этого, библиотека cuBool имеет пакет pycubool~\cite{net:pycubool} для работы в среде Python. Эти библиотеки были разработаны входе подобного исследования в 2021 году. Они могли бы быть взяты как основа в данной работе, однако архитектура решений не позволяет их расширить для использования произвольных типов и пользовательских функций.

\subsubsection*{Общие недостатки}

Далее выделены основные и наиболее важные недостатки представленных решений.

\begin{itemize}
    \item \textbf{Императивный интерфейс.} Стандарт GraphBLAS и многие сходные библиотеки имеют императивный интерфейс, где пользователь явно вызывает одну операцию за другой, используя соответствующие функции. Это существенно усложняет реализацию подобного интерфейса, а также вносит ряд ограничений на оптимизацию и распараллеливание, т.к. библиотека не обладает достаточной информацией о зависимостях между данными, а также о точках синхронизации.
    
    \item \textbf{Использование шаблонов.} Ряд библиотек, которые реализуют GraphBLAS в виде C++ API, используют шаблоны языка для обобщенного программирования примитивов и операций. Это снижает количество вспомогательного кода, однако делает невозможным единовременную компиляцию такой библиотеки и ее распространение для конечных пользователей без требования локальной компиляции. 
    
    \item \textbf{Использование Nvidia CUDA.} Библиотеки, использующие GPU-устройство для ускорения вычислений, полагаются чаще всего на Nvidia Cuda C++ API, так как оно имеет удобный механизм шаблонов. Однако Cuda технология поддерживается только на графических картах компании Nvidia, что существенно снижает количество потенциальных компьютеров для вычислений.
\end{itemize}

\subsection{Вычисления на GPU-устройстве}

\textit{GPGPU} (от англ. general-purpose computing on graphics processing units) --- техника использования графического процессора видеокарты компьютера для осуществления неспециализированных вычислений, которые обычно проводит центральный процессор. Данная техника позволяет получить значительный прирост производительности, когда необходимо обрабатывать большие массивы однородных данных фиксированным набором команд. 

Термин \textit{multi-GPU} (от англ. multiple GPUs) используется для обозначения вычислительной инфраструктуры, в которой доступно несколько GPU-устройств. Приложения, созданные для подобной среды, автоматически осуществляют распределение вычислений по всем доступным устройствам. 

Существует несколько промышленных стандартов для создания программ, использующих графический процессор, одними из которых являются Vulkan~\cite{net:spec_vulkan}, OpenGL~\cite{net:spec_opengl}, DirectX~\cite{net:spec_direct3d} как API для графических и неспециализированных вычислительных задач, а также OpenCL~\cite{net:spec_opencl}, Nvidia Cuda~\cite{net:cuda_toolkit_docs} как API для неспециализированных вычислений. 

В существующих графовых инструментах основными технологиями для распараллеливания вычислений на графических устройствах являются OpenCL и Nvidia Cuda. Далее представлен краткий обзор каждой из технологий.

\begin{itemize}
    \item \textbf{Nvida Cuda API} является проприетарной технологией компании Nvidia и доступен только на графических устройствах этой компании. Данный API имеет языковую поддержку как С так C++ возможностей, позволяет использовать шаблоны для обобщенного программирования, что упрощает разработку множества алгоритмов, таких как \textit{sort}, \textit{scan}, \textit{reduce}, которые параметризуются типами элементов и операциями. Кроме того, компания Nvidia поставляет множество инструментов для отладки и профилирования Cuda-кода. Также Nvidia Cuda имеет средства для multi-GPU программирования.
    
    \item \textbf{OpenCL API} это открытый стандарт для создания программ, использующих различного типа ускорители для распараллеливания вычислений. Данный стандарт имеет поддержку на множестве платформ, включая Intel, Nvidia, AMD, Apple M1, что делает его применимым на широком классе устройств. Данный API спроектирован в виде C-интерфейса, он не имеет встроенной поддержки для осуществления обобщенного программирования, как в Cuda C++. Также OpenCL имеет средства для multi-GPU программирования. Так как данный стандарт является открытым, на разных платформах качество его поддержки, а также актуальная версия отличаются, что делает проблематичным разработку и отладку OpenCL-приложений.
\end{itemize}

В данной работе в качестве технологии для программирования GPU-вычислений используется OpenCL. Данный API выбран, поскольку его требуемая версия 1.2 для проекта имеет поддержку на всех актуальных устройствах. Также данный API позволяет динамически во время исполнения компилировать код для выполнения на GPU, что делает его удобным инструментов для создания \textit{обобщенной} библиотеки, где пользователь сможет реализовывать свои примитивные типы и операции в виде набора текстовых строк.

% \subsection{Алгоритмы разреженной линейной алгебры для GPU}