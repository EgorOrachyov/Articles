\section*{Введение}

Данные, структурированные с использованием графовой модели, все чаще естественным образом появляются в реальных практических задачах, таких как вычисление запросов к базам данных~\cite{article:querying_graph_databases}, графовые базы данных~\cite{paper:redisgraph}, анализ социальных сетей~\cite{article:facebook_large_scale} анализ RDF данных~\cite{article:cfpq_and_rdf_analysis}, биоинформатика~\cite{article:rna_prediction} и статический анализ кода~\cite{article:dyck_cfl_code_analysis}. 

В графовой модели основные сущности представляются вершинами графа, а отношения между сущностями --- ориентированными ребрами с различными метками. Подобная модель позволяет относительно легко моделировать предметную область, сохраняя в явном виде множество сложных отношений между объектами, что не так просто выполнить, например, в классической \textit{реляционной модели}.

Достаточно больше количество реальных графовых данных имеет разреженную структуру, когда количество ребер сравнимо с количество вершин. Поэтому для решения прикладных задач наибольший интерес представляют инструменты, которые позволяю эффективно представлять в памяти такие данные. А поскольку граф может насчитывать десятки и сотни миллионов ребер~\cite{article:facebook_large_scale}, инструменты обработки графов должны предоставлять возможности для параллельной обработки с использованием нескольких вычислительных устройств или узлов.

За последние десятилетия исследовательское сообщество посвятило множество работ разработке различных инструментов для эффективного анализа  реальных графовых данных. Можно отметить такие проекты как Gunrock~\cite{article:gunrock}, Ligra~\cite{article:ligra},  GraphBLAST~\cite{yang2019graphblast}, SuiteSparse~\cite{article:suite_sparse_for_graph_problems} и многие другие. Существующие инструменты имеют различные стратегии распараллеливания вычислений, используют для ускорения операций графический процессор, ускоряют вычисления с использованием нескольких потоков центрального процессора. 

Как показывают различные исследования~\cite{yang2019graphblast, net:cubool_project}, использование графического процессора для ускорения вычислений является одним из перспективных направлений для получения высокопроизводительного анализа данных. Однако реализация прикладных алгоритмов и, в том числе, библиотек для анализа графов является очень трудоемкой задачей. Далее выделы основные причины, объясняющие эту проблему.

\begin{itemize}
    \item Программные интерфейсы для работы с GPU, такие как Nvidia CUDA и OpenCL, имеют низкоуровневую природу --- они многословны, требуют множества вспомогательной работы для осуществления простых операций, что делает их недоступными для рядового программиста.
   
    \item Многие алгоритмы на графах имеют сходную структуру, однако в деталях сильно отличаются, что требует точечной и аккуратной реализации каждого алгоритма с применением \textit{локальных} оптимизаций, что делает их неприменимыми для других алгоритмов.
   
    \item При использовании одного или нескольких GPU устройств, необходимо равномерно распределить работу между всеми вычислительными блоками, что проблематично в силу непредсказуемости нерегулярных шаблонов обращения к данным, используемых в графовых алгоритмах. 
   
    \item Графовые алгоритмы в основном требуют множество обращений к неструктурированным данным в памяти, когда количество непосредственных вычислений сохраняется низким. Поэтому необходимо использовать специализированные структуры хранения данных.
\end{itemize}

Чтобы ответить на множество возникающих вопросов, исследовательское сообщество предложило перспективную концепцию использования аппарата разреженной линейной алгебры для решения прикладных задач в виде стандарта GraphBLAS~\cite{paper:graphblas_foundations}, который предоставляет C API и позволяет выражать алгоритмы на графах в терминах операций над матрицами и векторами. Эффективная реализация только этих примитивов и операций позволяет получить готовую для вычислений реализацию алгоритма, не вдаваясь в делали низкоуровневого программирования.

Было представлено множество реализаций стандарта GraphBLAS, но полноценной его реализаций для вычислений на широком классе графических устройств в multi-GPU инфраструктуре на текущий момент не существует. Поэтому важной задачей является разработка библиотеки примитивов и операций разреженной линейной алгебры, предоставляющей возможность обобщения операций на произвольные пользовательские типы и операции, а также осуществляющей автоматическое распараллеливание вычислительных операций в multi-GPU инфраструктуре.   
