\section{Results}

The following results were achieved in this work.

\begin{itemize}
    \item The survey of the field was conducted. Model for a graph analysis were shown. Also, the concept of the linear algebra based approach was described in a great detail with a respect to a graph traversal and existing solutions. Introduction into a GraphBLAS standard was provided. Existing implementations, frameworks and most significant contributions for a graph analytic were studied. Their limitations were highlighted. General-purpose GPU computations concept was covered. Different APIs for GPU programming were presented. Their advantages and disadvantages were covered. General GPU programming challenges and pitfalls were highlighted. 
    
    \item The architecture of the library for a generalized sparse linear algebra for GPU computations was developed. The architecture and library design was based on a project requirements, as well as on a limitation and experience of the existing solutions. The differences with GraphBLAS standard were stated.
    
    \item The library was implemented accordingly to the developed architecture. The core of the library, interface, data containers, built-in scalar data types, built-in element-wise functions, expressions processing, OpenCL backend functionality, common linear algebra operations implementations were covered. Several graph algorithms for a graph analysis were implemented using developed library API.
    
    \item The preliminary experimental study of the proposed artifacts was conducted. Obtained results allowed to conclude, that the chosen method of the library development is a promising way to a high-performance graph analysis in terms of the linear algebra on a wide family of GPU devices. The proposed solution showed comparable performance to existing state-of-the-art solutions. The developed library showed a scalability on different device vendors GPUs. Also, the proposed solutions got an acceptable performance on integrated GPUs in some cases even if compared with highly-optimized multi-core CPU frameworks.
\end{itemize}

The library source code is published on a GitHub platform. It is available at \url{https://github.com/SparseLinearAlgebra/spla}.
