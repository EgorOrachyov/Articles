\section{Разработка библиотеки матричных операций}
 
\subsection{Мотивация}

Для эффективной реализации алгоритмов~\cite{inbook:kronecker_cfpq_adbis, inproceedings:matrix_cfpq} требуются высокопроизводительные библиотеки операций линейной алгебры. В качестве такой библиотеки для выполнения матричных операций на ЦПУ нередко используются \textit{SuiteSparse}~\cite{net:suite_sparse},  как эталонная реализация стандарта \textit{GraphBLAS}~\cite{net:graphblas}, который был разработан как некоторый инструмент для реализации алгоритмов обработки графов на языке линейной алгебры.  

Экспериментальное исследование~\cite{inproceedings:cfpq_matrix_evaluation} по реализации алгоритма Рустама Азимова~\cite{inproceedings:matrix_cfpq} на GPGPU с использованием операций над плотными булевыми матрицами показало, что вычисление на графическом процессоре дает значительный прирост производительности при обработке синтетических данных и данных среднего размера. Однако реальные графовые данные насчитывают порядка $10^7$ -- $10^9$ вершин и являются сильно разреженными, т.е. количество ребер в графе сравнимо с количеством вершин, поэтому плотные матрицы не подходят для обработки такого типа данных. 

Естественным решением было бы использование библиотеки линейной алгебры над разреженными матрицами, которая использовала бы для своих вычислений графический процессор системы. Библиотеки \textit{NVIDIA cuSPARSE}~\cite{net:cusparse_docs} и \textit{CUSP}~\cite{net:cusplibrary} для платформы NVIDIA CUDA предоставляют подобную функциональность, однако они имеют специализацию только для стандартных типов данных, таких как \textit{float}, \textit{double}, \textit{int} и \textit{long}. Для реализации алгоритмов~\cite{inbook:kronecker_cfpq_adbis, inproceedings:matrix_cfpq} требуются операции над разреженными булевыми матрицами, поэтому требуется специализация вышеуказанных библиотек для работы с булевыми примитивами. С одной стороны, библиотека \textit{cuSPARSE} имеет закрытый исходный код, что делает невозможным ее модификацию, с другой стороны, библиотека \textit{CUSP} имеет открытый исходный код и свободную лицензию, однако используемый ею алгоритм умножения разреженных матриц \textit{слишком} требователен к ресурсам памяти, что делает его неприменимым для обработки графовых данных большого размера.

В работе~\cite{inproceedings:cfqp_matrix_with_single_source} была предпринята попытка реализовать с нуля алгоритм~\cite{inproceedings:spgemm_mem_saving_for_nvidia} умножения разреженных матриц и специализировать его для конкретно булевых значений. Данный алгоритм эксплуатирует возможности  видеокарт от NVIDIA и за счет б\'ольшего времени на обработку позволяет снизить количество расходуемой видеопамяти. Эксперименты показали, что подобный подход позволяет не только снизить в разы количество расходуемой видеопамяти, но и снизить общее время работы алгоритма~\cite{inproceedings:matrix_cfpq} по сравнению с его реализацией на \textit{CUSP}. 

\subsection{Примитивы линейной алгебры}

\subsection{Описание реализации}