% По умолчанию используется шрифт 14 размера. Если нужен 12-й шрифт, уберите опцию [14pt]
\documentclass[14pt
  , russian
  %, xcolor={svgnames}
  ]{matmex-diploma-custom}
 
\usepackage[table]{xcolor}
\usepackage{graphicx}
\usepackage{tabularx}
\newcolumntype{Y}{>{\centering\arraybackslash}X}
\usepackage{amsmath}
\usepackage{amsthm}
\usepackage{amsfonts}
\usepackage{amssymb}
\usepackage{mathtools}
\usepackage{thmtools}
\usepackage{thm-restate}
\usepackage{tikz}
\usepackage{wrapfig}
% \usepackage[kpsewhich,newfloat]{minted}
% \usemintedstyle{vs}
\usepackage[inline]{enumitem}
\usepackage{subcaption}
\usepackage{caption}
\usepackage[nocompress]{cite}
\usepackage{makecell}

\usepackage{multirow}
\usepackage{graphicx}
\usepackage{subcaption}
\usepackage{lscape}
\usepackage{longtable}
\usepackage{tikz}
\usepackage{latexsym}

\usepackage{algpseudocode}
\usepackage{algorithm}
\usepackage{algorithmicx}
\usepackage{verbatim}
\usepackage{mathtools}

\usepackage{subcaption}
\usepackage{colortbl}
\usepackage{balance}

% \setitemize{noitemsep,topsep=0pt,parsep=0pt,partopsep=0pt}
% \setenumerate{noitemsep,topsep=0pt,parsep=0pt,partopsep=0pt}


\graphicspath{ {resources/} }

% 
% % \documentclass 
% %   [ a4paper        % (Predefined, but who knows...)
% %   , draft,         % Show bad things.
% %   , 12pt           % Font size.
% %   , pagesize,      % Writes the paper size at special areas in DVI or
% %                    % PDF file. Recommended for use.
% %   , parskip=half   % Paragraphs: noindent + gap.
% %   , numbers=enddot % Pointed numbers.
% %   , BCOR=5mm       % Binding size correction.
% %   , submission
% %   , copyright
% %   , creativecommons 
% %   ]{eptcs}
% % \providecommand{\event}{ML 2018}  % Name of the event you are submitting to
% % \usepackage{breakurl}             % Not needed if you use pdflatex only.
% 
% \usepackage{underscore}           % Only needed if you use pdflatex.
% 
% \usepackage{booktabs}
% \usepackage{amssymb}
% \usepackage{amsmath}
% \usepackage{mathrsfs}
% \usepackage{mathtools}
% \usepackage{multirow}
% \usepackage{indentfirst}
% \usepackage{verbatim}
% \usepackage{amsmath, amssymb}
% \usepackage{graphicx}
% \usepackage{xcolor}
% \usepackage{url}
% \usepackage{stmaryrd}
% \usepackage{xspace}
% \usepackage{comment}
% \usepackage{wrapfig}
% \usepackage[caption=false]{subfig}
% \usepackage{placeins}
% \usepackage{tabularx}
% \usepackage{ragged2e}
% \usepackage{soul}
\usepackage{csquotes}
% \usepackage{inconsolata}
% 
% \usepackage{polyglossia}   % Babel replacement for XeTeX
%   \setdefaultlanguage[spelling=modern]{russian}
%   \setotherlanguage{english}
% \usepackage{fontspec}    % Provides an automatic and unified interface 
%                          % for loading fonts.
% \usepackage{xunicode}    % Generate Unicode chars from accented glyphs.
% \usepackage{xltxtra}     % "Extras" for LaTeX users of XeTeX.
% \usepackage{xecyr}       % Help with Russian.
% 
% %% Fonts
% \defaultfontfeatures{Mapping=tex-text}
% \setmainfont{CMU Serif}
% \setsansfont{CMU Sans Serif}
% \setmonofont{CMU Typewriter Text}

\usepackage[final]{listings}

\lstdefinelanguage{ocaml}{
keywords={@type, function, fun, let, in, match, with, when, class, type,
nonrec, object, method, of, rec, repeat, until, while, not, do, done, as, val, inherit, and,
new, module, sig, deriving, datatype, struct, if, then, else, open, private, virtual, include, success, failure,
lazy, assert, true, false, end},
sensitive=true,
commentstyle=\small\itshape\ttfamily,
keywordstyle=\ttfamily\bfseries, %\underbar,
identifierstyle=\ttfamily,
basewidth={0.5em,0.5em},
columns=fixed,
fontadjust=true,
literate={->}{{$\to$}}3 {===}{{$\equiv$}}1 {=/=}{{$\not\equiv$}}1 {|>}{{$\triangleright$}}3 {\\/}{{$\vee$}}2 {/\\}{{$\wedge$}}2 {>=}{{$\ge$}}1 {<=}{{$\le$}} 1,
morecomment=[s]{(*}{*)}
}

\lstset{
mathescape=true,
%basicstyle=\small,
identifierstyle=\ttfamily,
keywordstyle=\bfseries,
commentstyle=\scriptsize\rmfamily,
basewidth={0.5em,0.5em},
fontadjust=true,
language=ocaml
}
 
\newcommand{\cd}[1]{\texttt{#1}}
\newcommand{\inbr}[1]{\left<#1\right>}


\newcolumntype{L}[1]{>{\raggedright\let\newline\\\arraybackslash\hspace{0pt}}m{#1}}
\newcolumntype{C}[1]{>{\centering\let\newline\\\arraybackslash\hspace{0pt}}m{#1}}
\newcolumntype{R}[1]{>{\raggedleft\let\newline\\\arraybackslash\hspace{0pt}}m{#1}}



\usepackage{soul}
\usepackage[normalem]{ulem}
%\sout{Hello World}



\begin{document}
%% Если что-то забыли, при компиляции будут ошибки Undefined control sequence \my@title@<что забыли>@ru
%% Если англоязычная титульная страница не нужна, то ее можно просто удалить.
\filltitle{ru}{
    %% Актуально только для курсовых/практик. ВКР защищаются не на кафедре а в ГЭК по направлению, 
    %%   и к моменту защиты вы будете уже не в группе.
    chair              = {Кафедра системного программирования},
    group              = {21.М07-мм},
    %% Макрос filltitle ненавидит пустые строки, поэтому обязателен хотя бы символ комментария на строке
    %% Актуально всем.
    title              = {Разработка библиотеки обобщенной разреженной линейной алгебры для вычислений на GPU},
    % 
    %% Здесь указывается тип работы. Возможные значения:
    %%   coursework - отчёт по курсовой работе;
    %%   practice - отчёт по учебной практике;
    %%   prediploma - отчёт по преддипломной практике;
    %%   practice - отчёт по учебной практике;
    %%   master - ВКР магистра;
    %%   bachelor - ВКР бакалавра.
    type               = {master},
    author             = {Орачев Егор Станиславович},
    % 
    %% Актуально только для ВКР. Указывается код и название направления подготовки. Типичные примеры:
    %%   02.03.03 <<Математическое обеспечение и администрирование информационных систем>>
    %%   02.04.03 <<Математическое обеспечение и администрирование информационных систем>>
    %%   09.03.04 <<Программная инженерия>>
    %%   09.04.04 <<Программная инженерия>>
    %% Те, что с 03 в середине --- бакалавриат, с 04 --- магистратура.
    specialty          = {09.04.04 <<Программная инженерия>>},
    % 
    %% Актуально только для ВКР. Указывается шифр и название образовательной программы. Типичные примеры:
    %%   СВ.5006.2017 <<Математическое обеспечение и администрирование информационных систем>>
    %%   СВ.5162.2020 <<Технологии программирования>>
    %%   СВ.5080.2017 <<Программная инженерия>>
    %%   ВМ.5665.2019 <<Математическое обеспечение и администрирование информационных систем>>
    %%   ВМ.5666.2019 <<Программная инженерия>>
    %% Шифр и название программы можно посмотреть в учебном плане, по которому вы учитесь. 
    %% СВ.* --- бакалавриат, ВМ.* --- магистратура. В конце --- год поступления (не обязательно ваш, если вы были в академе/вылетали).
    programme          = {ВМ.5666.2021 <<Программная инженерия>>},
    % 
    %% Актуально только для ВКР, только для матобеса и только 2017-2018 годов поступления. Указывается профиль подготовки, на котором вы учитесь.
    %% Названия профилей можно найти в учебном плане в списке дисциплин по выбору. На каком именно вы, вам должны были сказать после второго курса (можно уточнить в студотделе).
    %% Вот возможные вариканты:
    %%   Математические основы информатики
    %%   Информационные системы и базы данных
    %%   Параллельное программирование
    %%   Системное программирование
    %%   Технология программирования
    %%   Администрирование информационных систем
    %%   Реинжиниринг программного обеспечения
    % profile            = {Системное программирование},
    % 
    %% Актуально всем.
    supervisorPosition = {Доцент кафедры информатики, к.\,ф.-м.\,н.},
    supervisor         = {С.~В.~Григорьев},
    % 
    %% Актуально только для практик и курсовых. Если консультанта нет, закомментировать или удалить вовсе.
    % consultantPosition = {должность ООО <<Место работы>> степень},
    % consultant         = {К.К. Консультант},
    %
    %% Актуально только для ВКР.
    reviewerPosition   = {Эксперт, ООО "Техкомпания Хуавэй"},
    reviewer           = {С.~В.~Моисеев},
}

\filltitle{en}{
    chair              = {Software Engineering},
    group              = {21.М07-мм},
    title              = {Generalized sparse linear algebra library with vendor-agnostic GPUs accelerated computations},
    type               = {master},
    author             = {Egor Orachev},
    % 
    %% Possible choices:
    %%   02.03.03 <<Software and Administration of Information Systems>>
    %%   02.04.03 <<Software and Administration of Information Systems>>
    %%   09.03.04 <<Software Engineering>>
    %%   09.04.04 <<Software Engineering>>
    %% Те, что с 03 в середине --- бакалавриат, с 04 --- магистратура.
    specialty          = {09.04.04 <<Software Engineering>>},
    % 
    %% Possible choices:
    %%   СВ.5006.2017 <<Software and Administration of Information Systems>>
    %%   СВ.5162.2020 <<Programming Technologies>>
    %%   СВ.5080.2017 <<Software Engineering>>
    %%   ВМ.5665.2019 <<Software and Administration of Information Systems>>
    %%   ВМ.5666.2019 <<Software Engineering>>
    programme          = {СВ.5666.2021 <<Software Engineering>>},
    % 
    %% Possible choices:
    %%   Mathematical Foundations of Informatics
    %%   Information Systems and Databases
    %%   Parallel Programming
    %%   System Programming
    %%   Programming Technology
    %%   Information Systems Administration
    %%   Software Reengineering
    % profile            = {Software Engineering},
    % 
    %% Note that common title translations are:
    %%   кандидат наук --- C.Sc. (NOT Ph.D.)
    %%   доктор ... наук --- Sc.D.
    %%   доцент --- docent (NOT assistant/associate prof.)
    %%   профессор --- prof.
    supervisorPosition = {C.Sc., docent},
    supervisor         = {S.~V.~Grigorev},
    % 
    % consultantPosition = {position at ``Company'', degree if present},
    % consultant         = {C.C. Consultant},
    % %
    reviewerPosition   = {Expert, Huawei},
    reviewer           = {S.~V.~Moiseev},
}
\maketitle
\setcounter{tocdepth}{3}
\tableofcontents

% \begin{abstract}
%   В курсаче не нужен
% \end{abstract}

\section*{Введение}

Все чаще современные системы аналитики и рекомендаций строятся на основе анализа данных, структурированных с использованием \textit{графовой модели}. В данной модели основные сущности представляются вершинами графа, а отношения между сущностями --- ориентированными ребрами с различными метками. Подобная модель позволяет относительно легко и практически в явном виде моделировать сложные иерархические структуры, которые не так просто представить, например, в классической \textit{реляционной модели}. В качестве основных областей применения графовой модели можно выделить следующие: графовые базы данных~\cite{article:querying_graph_databases}, анализ RDF данных~\cite{article:cfpq_and_rdf_analysis}, биоинформатика~\cite{article:rna_prediction} и статический анализ кода~\cite{article:dyck_cfl_code_analysis}.

Поскольку графовая модель используется для моделирования отношений между объектами, при решении прикладных задач возникает необходимость в выявлении более сложных взаимоотношений между объектами. Для этого чаще всего формируются запросы в специализированных программных средствах для управления графовыми базами данных. В качестве запроса можно использовать некоторый \textit{шаблон} на путь в графе, который будет связывать объекты, т.е. выражать взаимосвязь между ними. В качестве такого шаблона можно использовать формальные грамматики, например, регулярные или контекстно-свободные (КС). Используя вычислительно более выразительные грамматики, можно формировать более сложные запросы и выявлять нестандартные и скрытые ранее взаимоотношения между объектами. Например, \textit{same-generation queries}~\cite{inbook:databases_intro}, сходные с сбалансированными скобочными последовательностями Дика, могут быть выражены КС грамматиками, в отличие от регулярных.

Результатом запроса может быть множество пар объектов, между которыми существует путь в графе, удовлетворяющий заданным ограничениям. Также может возвращаться один экземпляр такого пути для каждой пары объектов или итератор всех путей, что зависит от семантики запроса. Поскольку один и тот же запрос может иметь разную семантику, требуются различные программные и алгоритмические средства для его выполнения.  

Запросы с регулярными ограничениями изучены достаточно хорошо, языковая и программная поддержка выполнения подобных запросов присутствует в некоторых в современных графовых базах данных. Однако, полноценная поддержка запросов с КС ограничениями до сих пор не представлена. Существуют алгоритмы~\cite{article:cfpq_and_rdf_analysis, article:hellings_cfpq, inproceedings:matrix_cfpq, inbook:kronecker_cfpq_adbis, article:cfpq_go_for_rdf} для вычисления запросов с КС ограничениями, но потребуется еще время, прежде чем появиться полноценная высокпроизводительная реализация одного из алгоритмов, способная обрабатывать реальные графовые данные.

Работы~\cite{inproceedings:cfpq_matrix_evaluation, inproceedings:cfqp_matrix_with_single_source} в качестве реализации алгоритма~\cite{inproceedings:matrix_cfpq} для выполнения запросов с КС ограничениями с семантикой достижимости и семантикой одного пути показывают, что возможно использовать GPGPU для выполнения наиболее вычислительно сложных частей алгоритма, что дает \textit{существенный} прирост в производительности. 

Недавно представленный алгоритм~\cite{inbook:kronecker_cfpq_adbis} для вычисления запросов с КС ограничениями полагается на операции линейной алгебры: произведение Кронекера (частный случай тензорного произведения), умножение и сложение матриц в полукольце булевой алгебры. Важной задачей является реализация данного алгоритма, так как он в сравнении с~\cite{inproceedings:cfqp_matrix_with_single_source} позволяет выполнять запросы для всех ранее упомянутых семантик, потенциально поддерживает б\'ольшие по размеру КС запросы, с незначительными накладными расходами позволяет выполнять запросы с регулярными ограничениями, а с реализацией на GPGPU позволит получить потенциально приемлемое время выполнения запрсов.
\section{Problem statement}

The goal of this work is the implementation of the generalized sparse linear algebra primitives and operations library with portable vendor-agnostic yet high-performance GPUs accelerated computations. The work can be divided into the following tasks.

\begin{itemize}
    \item Conduct the survey of existing solutions, focusing on design principles and programming model, overview technologies and tools for programming GPU computations and highlight challenges of GPU programming.
    
    \item Develop the architecture of the library. Design the high-level library structure, execution model, storage scheme, GPUs backend for vendor-agnostic and portable computations acceleration.
    
    \item Implement the library according to the developed architecture, including library core, backend for GPUs accelerated computations, some GPU optimizations in order to speedup computations, and a set of common graph algorithms.

    % and a high-level package for library distribution.
    
    \item Conduct the preliminary experimental study of implemented artifacts. Analyse the performance of the proposed solution compared to existing tools, test the portability and scalability of the developed library on GPUs of different device vendors.
\end{itemize}
\section{Обзор предметной области}

\subsection{Терминология}

В этой секции изложены основные определения и факты из теории графов и формальных языков, необходимые для понимания предметной области. 
    
\textit{Ориентированный граф с метками} $\mathcal{G} = \langle V, E, L \rangle$ это тройка объектов, где $V$ конечное непустое множество вершин графа, $E \subseteq V \times L \times V$ конечное множество ребер графа, $L$ конечное множество меток графа. Здесь и далее будем считать, что вершины графа индексируются целыми числами, т.е. $V = \{0~...~|V| - 1\}$.

Граф $\mathcal{G} = \langle V, E, L \rangle$ можно представить в виде матрицы смежности $M$ размером $|V| \times |V|$, где $M[i,j] = \{l~|~(i,l,j) \in E\}$. Используя булеву матричную декомпозицию, можно представить матрицу смежности в виде набора матриц $\mathcal{M} = \{ M^l ~|~ l \in L, M^l[i,j] = 1 \iff l \in M[i,l]\}$.

Путь $\pi$ в графе $\mathcal{G} = \langle V, E, L \rangle$ это последовательность ребер $e_0,e_1,e_{n-1}$, где $e_i = (v_i, l_i, u_i) \in E$ и для любых $e_i, e_{i+1}: u_i = v_{i+1}$. Путь между вершинами $v$ и $u$ будем обозначать как $v \pi u$. Слово, которое формирует путь $\pi = (v_0, l_0, v_1), ... ,(v_{n-1}, l_{n-1}, v_n)$ будем обозначать как $\omega (\pi) = l_0 ... l_{n-1}$, что является конкатенацией меток вдоль этого пути $\pi$.

\textit{Контекстно-свободная (КС) грамматика} $G = \langle \Sigma, N, P, S \rangle$ это четверка объектов, где $\Sigma$ конечное множестве терминалов или алфавит, $N$ конечное множество нетерминалов, $P$ конечное множество правил вывода вида $A \rightarrow \gamma, \gamma \in (N \cup \Sigma)^*$, $S \in N$ стартовый нетерминал. 

Язык $L$ над конечным алфавитом символов $\Sigma$ --- множество всевозможных слов, составленных из символов этого алфавита, т.е. $L = \{\omega~|~w \in \Sigma ^*\}$.

\subsection{Поиск путей с ограничениями}

При вычислении запроса $p$ на поиск путей в графе $\mathcal{G} = \langle V, E, L \rangle$ в качестве ограничения выступает некоторый язык $L$, которому должны удовлетворять результирующие пути.

Поиск путей в графе с семантикой \textbf{достижимости}, это поиск всех таких пар вершин $(v,u)$, что между ними существует путь $v \pi u$ такой, что $\omega (\pi) \in L$. Результат запроса обозначается как $R = \{ (v,u)~|~\exists v \pi u : \omega (\pi) \in L \}$.

Поиск путей в графе с семантикой \textbf{всех путей}, это поиск всех таких путей $v \pi u$,   что $\omega (\pi) \in L$. Результат запроса обозначается как $\Pi = \{ v \pi u~|~v \pi u : \omega (\pi) \in L \}$.

Необходимо отметить, что множество $\Pi$ может быть бесконечным, поэтому в качестве результата запроса предполагается не всё множество в явном виде, а некоторый \textit{итератор}, который позволит последовательно извлекать все пути.

Семантика \textbf{одного пути} является ослабленной формулировкой семантики всех путей, так как для получения результата достаточно найти всего один путь вида $v \pi u : \omega (\pi) \in L$ для каждой пары $(v, u) \in R$.

Поскольку язык $L$ может быть бесконечным, при составлении запросов используют не множество $L$ в явном виде, а некоторое правило формирования слов этого языка. В качестве таких правил и выступают регулярные выражения или КС грамматики. При именовании запросов отталкиваются от типа правил, поэтому запросы именуются как регулярные или КС соответственно.

\subsection{Существующие решения}

Впервые проблема выполнения запросов с контекстно-свободными ограничениями была сформулирована в 1990 году в работе Михалиса Яннакакиса~\cite{inproceedings:yannakakis_cfpq_problem}. С того времени были представлены многие работы, в которых так или иначе предлагалось решение данной проблемы. Однако в недавнем исследовании Йохем Куиджперс и др.~\cite{article:kuijpers_cfpq_exp_compare} на основе сравнения нескольких алгоритмов~\cite{article:hellings_cfpq,inproceedings:matrix_cfpq,inbook:santos_cfpq_lr_analysis} для выполнения запросов с контекстно-свободными ограничениями заключили, что существующие алгоритмы неприменимы для анализа реальных данных в силу того, что обработка таких данных занимает значительное время. Стоит отметить, что алгоритмы, используемые в статье, были реализованы на языке программирования \textit{Java} и исполнялись в среде \textit{JVM} в однопоточном режиме, что не является сколь-угодно производительным решением.

Это подтверждают результаты работы~\cite{inproceedings:cfqp_matrix_with_single_source}, в которой с использование программных и аппаратных средств NVIDIA CUDA был реализован алгоритм Рустама Азимова~\cite{inproceedings:matrix_cfpq}. В данном алгоритме задача поиска путей с КС ограничениями для семантики одного пути сведена к операциям линейной алгебры, что позволяет использовать высокопроизводительные библиотеки для выполнения данных операций на GPGPU.

\subsection{Вычисления на GPGPU}

\textit{GPGPU} (от англ. General-purpose computing on graphics processing units) --- техника использования графического процессора видеокарты компьютера для осуществления неспециализированных вычислений, которые обычно проводит центральный процессор. Данная техника позволяет получить значительной прирост производительности, когда необходимо обрабатывать большие массивы данных с фиксированным набором команд по принципу \textit{SIMD}. 

Исторически видеокарты в первую очередь использовались как графические ускорители для создания высококачественной трехмерной графики в режиме реального времени. Однако, позже стало ясно, что мощность графического процессора можно использовать не только для графических вычислений. Так появились программируемые вычислительные блоки (англ. compute shaders), которые позволяют выполнять на видеокарте неграфические вычисления.

На данный момент существует несколько промышленных стандартов программирования вычислений на видеокарте, одними из которых являются Vulkan~\cite{net:spec_vulkan}, OpenGL~\cite{net:spec_opengl}, Direct3D~\cite{net:spec_direct3d} как API для преимущественно графических задач, а также OpenCL~\cite{net:spec_opencl}, NVIDIA CUDA~\cite{net:cuda_toolkit_docs} как API для неспециализированных вычислений.
\section{Поиск путей с КС ограничениями через произведение Кронекера}

Недавно представленный алгоритм~\cite{inbook:kronecker_cfpq_adbis} для выполнения КС запросов использует подобную технику сведения вычислений к операциям булевой алгебры: произведению Кронекера, матричному умножению и сложению. Однако структура алгоритма такова, что он позволяет выполнять запросы сразу в семантике достижимости и семантике всех путей, способен работать с КС грамматиками существенно большего размера, также имеет относительно небольшие накладные расходы при вычислении запросов с регулярными ограничениями, что делает его потенциально применимым для решения этого класса проблем. 

\subsection{Рекурсивный автомат}

Для представления входной грамматики КС запроса алгоритм~\cite{inbook:kronecker_cfpq_adbis} использует \textit{рекурсивный автомат}. Данный формализм является своего рода обобщением \textit{детерминированного конечного автомата} на случай КС языков. Для понимания того, как он устроен, обратимся к теории формальных языков.

\textit{Конечный автомат} (КА) $F = \langle \Sigma, Q, Q_s, Q_f, \delta \rangle$ это пятерка объектов, где $\Sigma$ конечное множество входных символов или алфавит, $Q$ конечное множество состояний, $Q_s \subseteq Q$ множество стартовых состояний, $Q_f \subseteq Q$ множество конечных состояний, $\delta : \Sigma \times Q \rightarrow 2^Q$ функция переходов автомата. Язык, допускаемый автоматом $F$ будем обозначать как $L(F)$. Любое регулярное выражение может быть преобразовано в соответствующий КА~\cite{book:automata_theory}. 

\textit{Рекурсивный автомат} (РА) $R = \langle M, m, \{C_i\}_{i \in M} \rangle$ это тройка объектов, где $M$ конечное множество меток компонентных КА, называемых далее \textit{модули}, $m$ метка стартового модуля, $\{C_i\}$ множество модулей, где модуль $C_i = \langle \Sigma \cup M, Q_i, S_i, F_i, \delta _i \rangle: $ состоит из:

\begin{itemize}
    \item $\Sigma \cup M$ множество символов модуля, $\Sigma \cap M = \emptyset$
    \item $Q_i$ конечное множество состояний модуля, $Q_i \cap Q_j = \emptyset, \forall i \neq j$
    \item $S_i \subseteq Q_i$ множество стартовых состояний модуля
    \item $F_i \subseteq Q_i$ множество конечных состояний модуля 
    \item $\delta_i : (\Sigma \cup M) \times Q_i \rightarrow 2^{Q_i}$ функция переходов
\end{itemize}

Рекурсивный автомат ведет себя как набор КА или модулей~\cite{article:recursive_state_machines}. Эти модули очень сходны с КА при обработке входных последовательностей символов, однако они способны обрабатывать дополнительные \textit{рекурсивные вызовы} за счет неявного \textit{стека вызовов}, который присутствует во время работы РА. С точки зрения прикладного программиста это похоже на рекурсивные вызовы одних функций из других с той разницей, что вместо функций здесь выступают модули РА.

Рекурсивные автоматы по своей вычислительной мощности эквивалентны автоматам на основе стека~\cite{article:recursive_state_machines}. А поскольку подобный стековый автомат способен распознавать КС грамматику~\cite{book:automata_theory}, рекурсивные автоматы эквивалентны КС грамматикам. Это позволяет корректно использовать РА для представления входной КС грамматики запроса.

\subsection{Пересечение рекурсивного автомата и графа}

Классический алгоритм~\cite{book:automata_theory} \textit{пересечения} двух КА $F^1$ и $F^2$ позволяет построить новый КА $F$ с таким свойством, что он допускает пересечение исходных регулярных языков, т.е. $L(F) = L(F^1) \cap L(F^2)$. Формально, для $F^1 = \langle \Sigma, Q^1, Q^1_S, Q^1_F, \delta^1 \rangle$ и $F^2 = \langle \Sigma, Q^2, Q^2_S, Q^2_F, \delta^2 \rangle$ строится новый КА $F = \langle \Sigma, Q, Q_S, Q_F, \delta \rangle$, где $Q = Q^1 \times Q^2$, $Q_S = Q^1_S \times Q^2_S$, $Q_F = Q^1_F \times Q^2_F$, $\delta: \Sigma \times Q \rightarrow Q$ и $\delta(s, \langle q_1, q_2 \rangle) = \langle q'_1, q'_2 \rangle$, если $\delta^1 (s, q_1)=q'_1$ и $\delta^2 (s, q_2)=q'_2$. 

Интерпретируя ориентированный граф с метками как некоторый конечный автомат, в котором все вершины графа являются одновременно начальными и конечными состояниями автомата, а ребра графа --- переходами между состояниями автомата, возможно пересечь этот граф и некоторый КА, используя алгоритм пересечения, описанный выше. Однако, если представить граф и функцию переходов КА, тоже интерпретируемую как граф, в виде матриц смежности, можно использовать \textit{произведение Кронекера} для построения функции переходов автомата пересечения.

\textit{Произведение Кронекера} для двух матриц $A$ и $B$ размера $m_1 \times n_1$ и $m_2 \times n_2$ с поэлементной операцией умножения $\cdot$ дает матрицу $M = A \otimes B$ размером $m_1 * m_2 \times n_1 * n_2$, где $M[u * m_2 + v, p * n_2 + q] = A[u, p] \cdot B[v, q]$. 

Поскольку РА состоит из набора модулей, которые по своей структуре не сильно отличаются от классических КА, это дает идею для применения похожего алгоритма пересечения РА и графа, с той  разницей, что процесс пересечения будет итеративным и будет включать в себя транзитивное замыкание, чтобы учесть \textit{рекурсивные вызовы}, присутствующие в РА. 

\subsection{Описание алгоритма}

В листинге~\ref{tensor:cfpq} представлен псевдокод алгоритма~\cite{inbook:kronecker_cfpq_adbis}. Необходимо отметить, что алгоритм использует булеву матричную декомпозицию в строках \textbf{3 -- 4} для представления матрицы переходов РА и матрицы смежности графа, а также использует матричное умножение, сложение и произведение Кронекера в строках \textbf{14 -- 16}.

Данный алгоритм является относительно простым и компактным, так как всю сложность выполнения он перекладывает на операции линейной алгебры, которые должны быть реализованы в сторонних высокопроизводительных библиотеках.

\begin{algorithm}[h]
\floatname{algorithm}{Listing}
\begin{algorithmic}[1]
\footnotesize
\caption{Поиск путей через произведение Кронекера}
\label{tensor:cfpq}
\Function{KroneckerProductBasedCFPQ}{G, $\mathcal{G}$}
    % Input data preparation
    \State{$R \gets$ Рекурсивный автомат для грамматики $G$}
    \State{$\mathcal{M}_1 \gets$ Матрица переходов $R$ в булевой форме}
    \State{$\mathcal{M}_2 \gets$ Матрица смежности $\mathcal{G}$ в булевой форме}
    \State{$C_3 \gets$ Пустая матрица}
    % Eps-transition handling for graph
    \For{$s \in \{0,...,dim(\mathcal{M}_1)-1\}$}
        \For{$S \in \textit{getNonterminals}(R,s,s)$}
            \For{$i \in \{0,...,dim(\mathcal{M}_2)-1\}$}
                % Or just $M_2^n[i,i] \gets M_2^n[i,i] \vee \{1\}$ ??? 
                \State{$M_2^S[i,i] \gets \{1\}$}
            \EndFor
        \EndFor
    \EndFor
    \While{Матрица смежности $\mathcal{M}_2$ изменяется}
        % Kronecker product (i.e. partial intersection)
        \State{$\mathcal{M}_3 \gets \mathcal{M}_1 \otimes \mathcal{M}_2$}
        \Comment{Вычисление произведения Кронекера}
        % Collapse to single Boolean matrix
        \State{$M'_3 \gets \bigvee_{M_3^a \in \mathcal{M}_3} M_3^a $}
        \Comment{Слияние матриц в одну булеву матрицу достижимости}
        % Closure over Boolean matrix only
        \State{$C_3 \gets \textit{transitiveClosure}(M'_3)$}
        \Comment{Транзитивное замыкание для учета рекурсивных вызовов}
        \State{$n \times n \gets$ dim($M_3)$}
        % Add non-terminals (possibly new)
        \For{$(i,j) \in \{0,...,n-1\} \times \{0,...,n-1\}$}
            \If{$C_3[i,j]$}
                \State{$s, f \gets \textit{getStates}(C_3,i,j)$}
                \State{$x, y \gets \textit{getCoordinates}(C_3,i,j)$}
                \For{$S \in \textit{getNonterminals}(R,s,f)$}
                    \State{$M_2^S[x,y] \gets \{1\}$}
                \EndFor
            \EndIf
        \EndFor
    \EndWhile
\State \Return $\mathcal{M}_2,C_3$
\EndFunction
\end{algorithmic}
\end{algorithm}

% \begin{algorithm}[]
% \floatname{algorithm}{Listing}
% \begin{algorithmic}[1]
% \footnotesize
% \caption{Вспомогательные функции для алгоритма поиска путей}
% \label{tensor:cfpq:help}
% \Function{GetStates}{$C, i, j$}
%     \State{$r \gets dim(\mathcal{M}_1)$}
%     \Comment{$\mathcal{M}_1$ матрица переходов в булевой форме для $R$}
%     \State \Return{$\left\lfloor{i / r}\right\rfloor, \left\lfloor{j / r}\right\rfloor$}
% \EndFunction
% \Function{GetCoordinates}{$C, i, j$}
%     \State{$n \gets dim(\mathcal{M}_2)$}
%     \Comment{$\mathcal{M}_2$ матрица смежности в булевой форме для $\mathcal{G}$}
%     \State \Return{$i \bmod n, j \bmod n$}
% \EndFunction
% \end{algorithmic}
% \end{algorithm}

% todo: paths extraction algorithm
% \subsection{Извлечение всех путей}
\section{Текущий прогресс}

Прогресс в работе на данный момент:

\begin{itemize}
    \item Выбран технологический стек для реализации библиотеки матричных примитивов: С/C++ для реализации интерфейса библиотеки и ее функциональности, CMake для сборки проекта, NVIDIA CUDA Toolkit 10 для написания кода, исполняемого на CUDA-совместимой видеокарте, NVIDIA Thrust для автоматизации работы с неуправляемыми ресурсами GPU
    \item Создан репозиторий проекта~\cite{net:cubool_project}, настроена автоматическая сборка с использованием инструментария \textit{Github Actions}. Добавлено описание проекта и инструкция для сборки.
    \item Создан C совместимый интерфейс для работы с примитивами библиотеки, а также добавлена непосредственно реализация интерфейса: создание и удаление матриц, запись и чтение значений матрицы, операции умножения, сложения, произведение Кронекера.
    \item Добавлен набор \textit{unit}-тестов для проверки корректности работы операций на основе сравнения с эталонной реализации тестируемых операций на ЦПУ.
    \item На языке Python с использованием библиотеки Ctypes реализован базовый уровень абстракции, необходимый для использования функции библиотеки матричных операций в тестовой инфраструктуре, которая также реализована на языке Python. Выбор Ctypes обусловлен тем, что данный модуль включен в стандартную поставку Python интерпретатора.
\end{itemize}

% \section{Эксперимент}
% Как мы проверяем, что  всё удачно получилось

% \subsection{Условия эксперимента}
% Железо (если актуально); входные данные, на которых проверяем наш подход; почему мы выбрали именно эти тесты

% \subsection{Исследовательские вопросы (Research questions)}
% Надо сформулировать то, чего мы хотели бы добиться работой (2 штуки будет хорошо):

% \begin{itemize}
% \item Хотим алгоритм, который лучше вот таких-то остальных
% \item Если в подходе можно включать/выключать составляющие, то насколько существенно каждая составляющая влияет на улучшения
% \item Если у нас строится приближение каких-то штук, то на сколько точными будут эти приближения
% \item и т.п.
% \end{itemize}

% \subsection{Метрики}

% Как мы сравниваем, что результаты двух подходов лучше или хуже
% \begin{itemize}
% \item Производительность
% \item Строчки кода
% \item Как часто алгоритм "угадывает" правильную классификацию входа
% \end{itemize}

% Иногда метрики вырожденные (да/нет), это не очень хорошо, но если в области исследований так принято, то ладно.

% \subsection{Результаты}
% Результаты понятно что такое. Тут всякие таблицы и графики

% В этом разделе надо также коснуться Research Questions.

% \subsubsection{RQ1} Пояснения
% \subsubsection{RQ2} Пояснения

% \subsection{Обсуждение результатов}

% Чуть более неформальное обсуждение, то, что сделано. Например, почему метод работает лучше остальных? Или, что делать со случаями, когда метод классифицирует вход некорректно.

% \section{Применение того, что сделано на практике (опциональный)}

% Если применение в лоб не работает, потому что всё изложено чуть более сжато и теоретично, надо рассказать тонкости и правильный метод применения результатов. 

% \section{Угрозы нарушения корректности (опциональный)}

% Если основная заслуга метода, это то, что он дает лучшие цифры, то стоит сказать, где мы могли облажаться, когда проводили численные замеры. 

% \section{Заключение}

% Кратко, что было сделано. Также здесь стоит писать задачи на будущее.

% \textbf{Для курсовых/дипломов.} Также стоит сделать список результатов, который будет 1 к одному соответствовать задачам из раздела~\ref{sec:task}.

% \begin{itemize}
% \item Результат к задаче 1 
% \item Результат к задаче 2
% \item и т.д.
% \end{itemize}

% \nocite{*}
\setmonofont[Mapping=tex-text]{CMU Typewriter Text}
\bibliographystyle{ugost2008ls}
\bibliography{kronecker_cfpq_gpu}

\end{document}