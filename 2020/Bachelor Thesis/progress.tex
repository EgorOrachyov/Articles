\section{Текущий прогресс}

Прогресс в работе на данный момент:

\begin{itemize}
    \item Выбран технологический стек для реализации библиотеки матричных примитивов: С/C++ для реализации интерфейса библиотеки и ее функциональности, CMake для сборки проекта, NVIDIA CUDA Toolkit 10 для написания кода, исполняемого на CUDA-совместимой видеокарте, NVIDIA Thrust для автоматизации работы с ресурсами GPU.
    \item Создан репозиторий проекта~\cite{net:cubool_project}, настроена автоматическая сборка с использованием инструментария \textit{Github Actions}. Добавлено описание проекта и инструкция для сборки.
    \item Создан C совместимый интерфейс для работы с примитивами библиотеки, а также добавлена непосредственно реализация интерфейса: создание и удаление матриц, запись и чтение значений матрицы, операции умножения, сложения, произведение Кронекера.
    \item Добавлен набор \textit{unit}-тестов для проверки корректности работы операций на основе сравнения с эталонной реализации тестируемых операций на центральном процессоре.
    \item На языке Python с использованием библиотеки Ctypes реализован базовый уровень абстракции, необходимый для использования функции библиотеки матричных операций в тестовой инфраструктуре, которая также реализована на языке Python.
\end{itemize}