% Тут используется класс, установленный на сервере Papeeria. На случай, если
% текст понадобится редактировать где-то в другом месте, рядом лежит файл matmex-diploma-custom.cls
% который в момент своего создания был идентичен классу, установленному на сервере.
% Для того, чтобы им воспользоваться, замените matmex-diploma на matmex-diploma-custom
% Если вы работаете исключительно в Papeeria то мы настоятельно рекомендуем пользоваться
% классом matmex-diploma, поскольку он будет автоматически обновляться по мере внесения корректив
%

% По умолчанию используется шрифт 14 размера. Если нужен 12-й шрифт, уберите опцию [14pt]
%\documentclass[14pt]{matmex-diploma}
\documentclass[14pt]{matmex-diploma-custom}
\usepackage{multirow}
\usepackage{multirow}
\usepackage[table,xcdraw]{xcolor}
\usepackage{graphicx}
\usepackage{subcaption}
\usepackage{lscape}
\usepackage{longtable}
\usepackage{tikz}


\begin{document}
% Год, город, название университета и факультета предопределены,
% но можно и поменять.
% Если англоязычная титульная страница не нужна, то ее можно просто удалить.
\filltitle{ru}{
    chair              = {Программная инженерия\\ Кафедра системного программирования},
    title              = {Реализация алгоритма поиска путей в графовых базах данных через тензорное произведение на GPGPU},
    % Здесь указывается тип работы. Возможные значения:
    %   coursework - Курсовая работа
    %   diploma - Диплом специалиста
    %   master - Диплом магистра
    %   bachelor - Диплом бакалавра
    type               = {bachelor},
    position           = {студента},
    group              = 471,
    author             = {Орачев Егор Станиславович},
    supervisorPosition = {к.\,ф.-м.\,н., доцент},
    supervisor         = {С. В. Григорьев},
%   reviewerPosition   = {ст. преп.},
%   reviewer           = {Привалов А.\,И.},
%   chairHeadPosition  = {д.\,ф.-м.\,н., профессор},
%   chairHead          = {Хунта К.\,Х.},
    university         = {Санкт-Петербургский Государственный Университет},
%   faculty            = {Математико-механический факультет},
    city               = {Санкт-Петербург},
    year               = {2020}
}

% \filltitle{en}{
%     chair              = {The Meaning of Life \\ Uselessness of Everything},
%     title              = {Empty subset as closed set},
%     author             = {Edelweis Mashkin},
%     supervisorPosition = {professor},
%     supervisor         = {Amvrosy Vibegallo},
%     reviewerPosition   = {assistant},
%     reviewer           = {Alexander Privalov},
%     chairHeadPosition  = {professor},
%     chairHead          = {Christobal Junta},
% }

\maketitle
\tableofcontents
% У введения нет номера главы
\section*{Введение}

Все чаще современные системы аналитики и рекомендаций строятся на основе анализа данных,     
структурированных с использованием \textit{графовой модели}. В данной модели основные сущности 
представляются вершинами графа, а отношения между сущностями --- ориентированными ребрами с 
различными метками. Подобная модель позволяет относительно легко и практически в явном виде 
моделировать сложные иерархические структуры, которые не так просто представить, например, в 
классической \textit{реляционной модели}. В качестве основных областей применения графовой 
модели можно выделить следующие: графовые базы данных~\cite{article:querying_graph_databases}, 
анализ RDF данных~\cite{DBLP:journals/corr/ZhangFWR15}, 
биоинформатика~\cite{article:rna_prediction} и статический анализ 
кода~\cite{article:dyck_cfl_code_analysis}.

Поскольку графовая модель используется для моделирования отношений между объектами, при решении 
прикладных задач возникает необходимость выявления более сложных взаимоотношений между 
объектами. Для этого чаще всего формируются запросы в специализированных программных средствах 
для управления графовыми базами данных. В качестве запроса можно использовать некоторый 
\textit{шаблон} на путь в графе, который будет связывать объекты, т.е. выражать взаимосвязь 
между ними. В качестве такого шаблона можно использовать формальные грамматики, например, 
регулярные или контекстно-свободные (КС). Используя вычислительно более выразительные 
грамматики, можно формировать более сложные запросы и выявлять нестандартные и скрытые ранее 
взаимоотношения между объектами. Например, \textit{same-generation 
queries}~\cite{inbook:databases_intro}, сходные с сбалансированными скобочными 
последовательностями Дика, могут быть выражены КС грамматиками, в отличие от регулярных.

Результатом запроса может быть множество пар объектов, между которыми существует путь в графе, 
удовлетворяющий заданным ограничениям. Также может возвращаться один экземпляр такого пути для 
каждой пары объектов или итератор всех путей, что зависит от семантики запроса. Поскольку один 
и тот же запрос может иметь разную семантику, требуются различные программные и алгоритмические
средства для его выполнения.  

Запросы с регулярными ограничениями изучены достаточно хорошо, языковая и программная поддержка 
выполнения подобных запросов присутствует в некоторых в современных графовых базах данных. 
Однако, полноценная поддержка запросов с КС ограничениями до сих пор не представлена. Существуют
алгоритмы~\cite{DBLP:journals/corr/ZhangFWR15, article:hellings_cfpq, inproceedings:matrix_cfpq,
inbook:kronecker_cfpq_adbis, article:cfpq_go_for_rdf} для вычисления запросов с КС 
ограничениями, но потребуется еще время, прежде чем появиться полноценная высокпроизводительная 
реализация одного из алгоритмов, способная обрабатывать реальные графовые данные.

Работы~\cite{inproceedings:cfpq_matrix_evaluation, inproceedings:cfqp_matrix_with_single_source}
в качестве реализации алгоритма~\cite{inproceedings:matrix_cfpq} показывают, что возможно 
использовать GPGPU для выполнения наиболее вычислительно сложных частей алгоритма, что дает 
\textit{существенный} прирост в производительности. Недавно представленный 
алгоритм~\cite{inbook:kronecker_cfpq_adbis} для вычисления запросов с КС ограничениями 
полагается на операции линейной алгебры, в частности, произведение Кронекера (частный случай 
тензорного произведения), умножение и сложение матриц в полукольце булевой алгебры. Данный 
алгоритм позволяет выполнять запросы для всех ранее упомянутых семантик, потенциально 
поддерживает больш\'ие по размеру КС запросы, а также хорошо реализуется с помощью программных 
средств для вычисления на GPGPU.

Таким образом, важной задачей является не только реализация перспективного 
алгоритма~\cite{inbook:kronecker_cfpq_adbis} для выполнения запросов с КС ограничениям,
но и разработка программной библиотеки для работы с примитивами линейной булевой алгебры, 
которая позволила бы упростить прототипирование и реализацию подобного и будущих алгоритмов на 
GPGPU, в частности, на платформе NVIDIA CUDA~\cite{net:cuda_toolkit_docs}.

\section{Постановка задачи}

Цель данной работы --- реализация алгоритма поиска путей в графовых базах данных через тензорное
произведение на платформе NVIDIA CUDA в качестве GPGPU технологии. Для ее достижения были 
поставлены следующие задачи:

\begin{itemize}
    \item Реализация библиотеки для работы с примитивами булевой алгебры на GPGPU
    \item Реализация алгоритма поиска путей
    % \item Реализация алгоритма восстановления путей
    \item Экспериментальное исследование реализации алгоритма
\end{itemize}

\section{Обзор предметной области}

\subsection{Терминология}

\subsection{Поиск путей с контекстно-свободными ограничениями}

\subsection{Существующие решения}

\subsection{Поиск путей через произведение Кронекера}

\section{Реализация библиотеки матричных операций}

\section{Реализация алгоритма}

\section{Экспериментальное исследование}

% У заключения нет номера главы
\section*{Заключение}

\setmonofont[Mapping=tex-text]{CMU Typewriter Text}
\bibliographystyle{ugost2008ls}
\bibliography{diploma.bib}
\end{document}
